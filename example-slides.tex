\documentclass[t,german]{beamer}

%%
%% Wir benutzen das beamer-theme 'hsbo'
%%
\usetheme[numbering=fullbar]{hsbo}
\usepackage{tikz}

\title{Beispiel-Titel}
\subtitle{Einführung in das HSBO-Thema}

\author{Dr. Christian Bockermann}
\institute{Hochschule Bochum}
\date{Semester Year/Year+1}

\begin{document}

%%
%% Titel-Folie
%%
\begin{frame}[fragile]
\maketitle
\end{frame}

\section{Beispiel-Foliensatz}

%%
%% Beispiel - Folie-1
%%
\begin{frame}{Folie Titel-1}
\textbf{\alert{Beispiele}} sind wichtig für das Verständnis

\begin{itemize}
	\item Sie erklären anschaulich, wie Dinge in der Praxis laufen
	\item Sie geben einen schnellen Einstieg zum eigenen Handeln
\end{itemize}
\end{frame}


\begin{frame}{Folie - Farben}

\textbf{Das Thema HSBO definiert ein paar \alert{Farben}}:

\medskip

\begin{center}
\begin{tikzpicture}
\tikzstyle{box}=[inner sep=0pt,minimum width=0.5cm,minimum height=0.5cm,text width=0cm]
\node[box,fill=hsbo] at (0,0) {};
\node[anchor=west] at (0.75,0) {Die Farbe {\ttfamily \textbackslash color\{hsbo\}}};

\node[box,fill=blau0] at (0,-0.75) {};
\node[anchor=west] at (0.75,-0.75) {Die Farbe {\ttfamily \textbackslash color\{blau0\}}};

\node[box,fill=blau1] at (0,-1.5) {};
\node[anchor=west] at (0.75,-1.5) {Die Farbe {\ttfamily \textbackslash color\{blau1\}}};

\node[box,fill=blau2] at (0,-2.25) {};
\node[anchor=west] at (0.75,-2.25) {Die Farbe {\ttfamily \textbackslash color\{blau2\}}};

\node[box,fill=grau0] at (0,-3) {};
\node[anchor=west] at (0.75,-3) {Die Farbe {\ttfamily \textbackslash color\{grau0\}}};

\node[box,fill=grau1] at (0,-3.75) {};
\node[anchor=west] at (0.75,-3.75) {Die Farbe {\ttfamily \textbackslash color\{grau1\}}};


\end{tikzpicture}
\end{center}

Der {\ttfamily \textbackslash alert\{some text\}} Befehl verwendet die Farbe {\ttfamily hsbo}

\end{frame}

\end{document}