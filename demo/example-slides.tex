\documentclass[t,german]{beamer}

%%
%% Wir benutzen das beamer-theme 'hsbo'
%%
\usetheme[numbering=fullbar]{hsbo}
%\usecolortheme{hsbo}
\usepackage{tikz,verbatimbox}


\title{Beispiel-Titel}
\subtitle{Einführung in das HSBO-Thema}

\author{Dr. Christian Bockermann}
\institute{Hochschule Bochum}
\date{Semester Year/Year+1}

\begin{document}
\begin{myverbbox}{\exampleSlide}
\documentclass{beamer}
\usetheme{hsbo}

\begin{document}

\begin{frame}{Folie Titel-1}
  \textbf{\alert{Beispiele}} sind wichtig für das Verständnis
  \begin{itemize}
  	 \item Sie erklären anschaulich, wie Dinge in der Praxis laufen
    \item Sie geben einen schnellen Einstieg zum eigenen Handeln
  \end{itemize}
\end{frame}

\end{document}
\end{myverbbox}


%%
%% Titel-Folie
%%
\begin{frame}[fragile]
\maketitle
\end{frame}

\section{Beispiel-Foliensatz}

%%
%% Beispiel - Folie-1
%%
\begin{frame}[fragile]{Folie Titel-1}
\textbf{\alert{Beispiele}} sind wichtig für das Verständnis

\begin{itemize}
	\item Sie erklären anschaulich, wie Dinge in der Praxis laufen
	\item Sie geben einen schnellen Einstieg zum eigenen Handeln
\end{itemize}

\begin{center}
\begin{tikzpicture}
\node[scale=0.7,anchor=west] at (0,0) {\exampleSlide};

\onslide<2->{
	\draw[draw=hsbo,thick] (-0.125,1.65) rectangle (5,2.35);
	\node[anchor=west,text width=4cm,align=center,scale=0.75] at (5.5,2) {{\textbf{Ein {\ttfamily beamer} Dokument mit Thema {\ttfamily hsbo}}}};
	\draw[hsbo,thick] (5,2) -- (5.5,2);
}

\end{tikzpicture}
\end{center}
\end{frame}



\begin{frame}{Folie - Farben}

\textbf{Das Thema HSBO definiert ein drei \alert{Grundfarben}}:

\begin{center}
\begin{tikzpicture}
\tikzstyle{box}=[inner sep=0pt,minimum width=0.5cm,minimum height=0.5cm,text width=0cm]
\node[box,fill=hsbo] at (0,0) {};
\node[anchor=west] at (0.75,0) {Die Farbe {\ttfamily \textbackslash color\{rot\}} = {\ttfamily \textbackslash color\{hsbo\}}};

\node[box,fill=blau] at (0,-0.75) {};
\node[anchor=west] at (0.75,-0.75) {Die Farbe {\ttfamily \textbackslash color\{blau\}}};

\node[box,fill=grau] at (0,-1.5) {};
\node[anchor=west] at (0.75,-1.5) {Die Farbe {\ttfamily \textbackslash color\{grau\}}};

% \node[box,fill=blau2] at (0,-2.25) {};
% \node[anchor=west] at (0.75,-2.25) {Die Farbe {\ttfamily \textbackslash color\{blau2\}}};

% \node[box,fill=grau0] at (0,-3) {};
% \node[anchor=west] at (0.75,-3) {Die Farbe {\ttfamily \textbackslash color\{grau0\}}};

% \node[box,fill=grau1] at (0,-3.75) {};
% \node[anchor=west] at (0.75,-3.75) {Die Farbe {\ttfamily \textbackslash color\{grau1\}}};

% \node[box,fill=grau2] at (0,-4.5) {};
% \node[anchor=west] at (0.75,-4.5) {Die Farbe {\ttfamily \textbackslash color\{grau2\}}};

\end{tikzpicture}
\end{center}

Die Abstufungen sind jeweils {\ttfamily farbe}={\ttfamily farbe0}, {\ttfamily farbe1}, {\ttfamily farbe2}.

\smallskip

Die Farbe {\ttfamily hsbo} ist dabei ein Alias für {\ttfamily rot}


\medskip

Der {\ttfamily \textbackslash alert\{some text\}} Befehl verwendet die Farbe {\ttfamily hsbo}
\end{frame}


\begin{frame}{Folien - Farben 2}

Die drei Hauptfarben {\ttfamily rot} (={\ttfamily hsbo}), {\ttfamily blau} und {\ttfamily grau}.

\begin{center}
\begin{tikzpicture}[scale=0.75,transform shape]

\draw[black!20,ultra thick] (90:2cm) -- (0,0);
\draw[black!20,ultra thick] (210:2cm) -- (0,0);
\draw[black!20,ultra thick] (330:2cm) -- (0,0);

\fill[rot] (90:2cm) circle (2ex);
\node at (90:3cm) {\ttfamily rot};

\fill[rot1] (120:2cm) circle (2ex);
\node at (120:3cm) {\ttfamily rot1};

\fill[rot2] (150:2cm) circle (2ex);
\node at (150:3cm) {\ttfamily rot2};

\fill[blau] (210:2cm) circle (2ex);
\node at (210:3cm) {\ttfamily blau};

\fill[blau1] (240:2cm) circle (2ex);
\node at (240:3cm) {\ttfamily blau1};

\fill[blau2] (270:2cm) circle (2ex);
\node at (270:3cm) {\ttfamily blau2};

\fill[grau] (330:2cm) circle (2ex);
\node at (330:3cm) {\ttfamily grau};

\fill[grau1] (360:2cm) circle (2ex);
\node[] at (360:3cm) {\ttfamily grau1};

\fill[grau2] (30:2cm) circle (2ex);
\node at (30:3cm) {\ttfamily grau2};
\end{tikzpicture}
\end{center}

\end{frame}

%%
%% sample tikz picture
%%
\begin{frame}{Folie - Farben 3}
Ein Beispiel Scatter-Plot mit den Hauptfarben:
\begin{center}
\begin{tikzpicture}[scale=0.65,transform shape]
\draw[->,thick] (0,0) -- (10,0);
\draw[->,thick] (0,0) -- (0,7);



% \fill[hsbo] (1,-1) rectangle (0.5,-0.5);
\node[rectangle,fill=hsbo,inner sep=0pt,text width=0cm,minimum height=0.25cm,minimum width=0.25cm] at (1,-0.5) {};
\node[anchor=west] at (1.25,-0.5) {Zufall 1};

\node[rectangle,fill=blau,inner sep=0pt,text width=0cm,minimum height=0.25cm,minimum width=0.25cm] at (3.5,-0.5) {};
\node[anchor=west] at (3.75,-0.5) {Zufall 2};

\node[rectangle,fill=grau,inner sep=0pt,text width=0cm,minimum height=0.25cm,minimum width=0.25cm] at (6,-0.5) {};
\node[anchor=west] at (6.25,-0.5) {Zufall 3};
\end{tikzpicture}
\end{center}
\end{frame}




\begin{frame}[fragile]{VBA Code}
Ein Beispiel für VBA Code in Farbe:
\begin{vba}
Sub ConvertToValues()
  With ActiveSheet.UsedRange.Value = .Value
  End With
End Sub
\end{vba}

\end{frame}


%%
%%
%%
\begin{myverbbox}{vbaCode}
\begin{frame}[fragile]{VBA Code}
Ein Beispiel für VBA Code in Farbe:
\begin{vba}
  Sub ConvertToValues()
    With ActiveSheet.UsedRange.Value = .value
    End With
  End Sub
\end{vba}
\end{frame}
\end{myverbbox}

\begin{frame}[fragile]{VBA Code - TeX Code}
Die vorherige Folie mit VBA Code wurde definiert durch:

\begin{center}
\begin{tikzpicture}
\node[scale=0.75] at (0,0) {\vbaCode};
\end{tikzpicture}
\end{center}
Die {\ttfamily \textbackslash begin\{vba\} .. \textbackslash end\{vba\}} Umgebung
übernimmt die Formatierung des VBA Codes
\end{frame}

\begin{myverbbox}{pyExample}
\begin{frame}[fragile]{Python Code}
  \begin{python}
     ...
  \end{python}
\end{frame}
\end{myverbbox}

\begin{frame}[fragile]{Python Code}
Auf die gleiche Weise steht mit der \alert{python} Umgebung auch eine
Möglichkeit für Python Code zur Verfügung:

\smallskip
\begin{python}
def func(x):
	"""Beispiel Funktion mit Doc String"""
	return x**3 + 2*x**2 - 17
\end{python}

\medskip

Wichtig ist der Zusatz {\color{blau0}\ttfamily [fragile]} bei dem umgebenden {\ttfamily frame}, da sonst die Code Umgebungen nicht funktionieren:

\begin{center}
\begin{tikzpicture}[]
	\node at (0,0) {\pyExample};
\end{tikzpicture}
\end{center}
\end{frame}
\end{document}